%\section*{LITERATURE CITED}
% extra space between lines in TOC
\addtocontents{toc}{\protect\vspace{1.0em}\protect}
\addcontentsline{toc}{section}{LITERATURE CITED}

%----------------------------------------------------------------------
%  Bibliography
%----------------------------------------------------------------------
\begin{thebibliography}{}

\bibitem[{\textit{Box}(1976)}]{Box1976}
Box, G. E. P. (1976), Science and Statistics, \textit{Journal of the American Statistical Association} \textit{71}, 791–799.

\bibitem[{\textit{Corbett et al.}(1999)}]{Corbett1999}
Corbett D. R., Chanton J., Burnett W., Dillon K., Rutkowski, C., Fourqurean, J. W. (1999), Patterns of groundwater discharge into Florida Bay, \textit{Limnol. Oceanogr.}, \textit{44}(4), 1045--1055. 

\bibitem[{\textit{Cosby et al.}(2010)}]{Cosby2010}
Cosby B., Marshall F. and Nuttle W., (2010), FATHOM (Version 6.10) Model Structure and Salinity Simulation, 59 p. 

\bibitem[{\textit{Hittle et al.}(2001)}]{Hittle2001}
Hittle C., Patino E., and Zucker, M. A., (2001), Freshwater flow from estuarine creeks into northeastern Florida Bay, U.S. Geological Survey Water-Resources Investigations Report 2001-4164, 32 p. https://pubs.er.usgs.gov/publication/wri014164

\bibitem[{\textit{Kelble et al.}(2007)}]{Kelble2007}
Kelble, C.R., Johns, E.M., Nuttle, W.K., Lee, T.N., Smith, R.H., Ortner, P.B. (2007), Salinity patterns in Florida Bay, \textit{Estuarine, Coastal and Shelf Science}, \textit{71}, 318--334. doi:10.1016/j.ecss.2006.08.006.

\bibitem[{\textit{Langevin et al.}(2004)}]{Langevin2004}
Langevin, C. D., Swain, E. D., and Wolfert, M. A. (2004), Simulation of Integrated Surface-Water/Ground-Water Flow and Salinity for a Coastal Wetland and Adjacent Estuary, U.S. Geological Survey Open-File Report 2004-1097, 30 p. http://pubs.usgs.gov/of/2004/1097/

\bibitem[{\textit{Marshall et al.}(2008)}]{Marshall2008}
Marshall, F.E, Smith, D.T and Nuttle W. (2008), Simulating and forecasting salinity in Florida Bay: A review of models, Cooperative Agreement Number CA H5284-05-0006 Between The United States Department of the Interior National Park Service Everglades National Park and Cetacean Logic Foundation, Inc. http://sofia.usgs.gov/publications/reports/salinity\_flbay/salinity\_models.pdf

\bibitem[{\textit{Marshall et al.}(2011)}]{Marshall2011}
Marshall, F.E, Smith, D.T and D.M. Nickerson (2011), Empirical tools for simulating salinity in the estuaries in Everglades National Park, Florida. \textit{Estuarine, Coastal and Shelf Science} \textit{95}, 377--387. doi:10.1016/j.ecss.2011.10.001.

\bibitem[{\textit{Tabb}(1967)}]{Tabb1967}
Tabb, D.C., (1967), Predictions of Estuarine Salinities in Everglades National Park, Florida, by the Use of Ground Water Records, Ph.D. dissertation, 107 pp., University of Miami, Coral Gables, Florida.

\bibitem[{\textit{Telis et al.}(2014)}]{Telis2014}
Telis, P.A., Xie, Zhixiao, Liu, Zhongwei, Li, Yingru, and Conrads, P.A., (2015), The Everglades Depth Estimation Network (EDEN) Surface-Water Model, Version 2: U.S. Geological Survey Scientific Investigations Report 2014-5209, 42 p., doi 10.3133/sir20145209.  http://sofia.usgs.gov/eden/

\end{thebibliography}


%\begingroup

%\normalsize
%\leftskip=2em
%\parindent=-2em

% this verbatim section is to trick latex into (reverse) indenting the first line of the section
%\begin{verbatim}
%\end{verbatim}

%%%TechReport{Anderson2010,
%Anderson G.H., K.E. Bahm, R.J. Fennema, E.D. Swain, K.M. Balentine, and T.J. Smith III. 2010. A paired surface-water/groundwater monitoring network in the western coastal mangrove Everglades provides water level and salinity data for analysis and model validation [abstract]. In: GEER 2010, Greater Everglades Ecosystem Restoration, The Everglades: A Living Laboratory of Change, Planning, Policy and Science, Naples, FL, July 12--16, 2010, University of Florida and U.S. Geological Survey, p. 7. [URL: \url{http://www.conference.ifas.ufl.edu/GEER2010/pdf/Abstract\%20BOOK.pdf}]

%\endgroup

\clearpage
